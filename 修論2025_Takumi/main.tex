\documentclass[a4j, 12pt]{article}
\usepackage[dvipdfmx]{color}
\usepackage[dvipdfmx]{graphicx}
\usepackage{iftex}
\ifPDFTeX
    % pLaTeX-only font packages are skipped under pdfLaTeX
\else
    \usepackage[ipaex]{pxchfon}
    \usepackage{otf}
\fi
\usepackage[utf8]{inputenc}
\usepackage{ascmac}
\usepackage{booktabs}
\usepackage{amsfonts}
\usepackage{amsmath}
\usepackage{amsthm}
\usepackage{amssymb}
\usepackage{appendix}

% Theorem environments (italic text)
\newtheorem{theorem}{Theorem}[section]
\newtheorem{lemma}[theorem]{Lemma}
\newtheorem{proposition}[theorem]{Proposition}
\newtheorem{corollary}[theorem]{Corollary}

% Definition environments (upright text)
\theoremstyle{definition}
\newtheorem{definition}[theorem]{Definition}
\newtheorem{example}[theorem]{Example}

% Remark environments (upright text, smaller spacing)
\theoremstyle{remark}
\newtheorem{remark}[theorem]{Remark}

\usepackage{comment}
\usepackage{enumerate}
\usepackage{here}
\usepackage{latexsym}
\usepackage{multirow}
\usepackage{subfigure}
\usepackage{url}
\usepackage{threeparttable}
% \usepackage{caption}
% \captionsetup[table]{justification=centering}
% \captionsetup[figure]{justification=centering}

\usepackage{sakolab-thesis}

%
% 論文の表紙の項目
%
%\thesistype{令和6年度 修士論文}
\date{, 2026}
\title{Enhancing the Flexibility and Automation of Post-Quantum Anonymous Credentials: A Comparative Analysis of Zero-Knowledge Virtual Machines and SNARK Circuit Compilers}
% todo: 英語版タイトルの変更
\etitle{A Thesis Submitted to the Department of Computer Science and Communications Engineering, the Graduate School of Fundamental Science and Engineering of Waseda University in Partial Fulfillment of the Requirements for the Degree of
Master of Engineering}
\titlesize{24truept}
\department{\fontsize{17pt}{17pt}\selectfont The Department of Computer Science and Communications Engineering, the }
\major{\fontsize{17pt}{17pt}\selectfont Graduate School of Fundamental Science and Engineering of Waseda University}
\guidance{Research on Cryptographic Protocols}
\supervisor{Prof. Kazue Sako}
\studentid{5124FG15-6}
\author{Takumi Otsuka}

\begin{document}
    %表紙
    \maketitle
    
    %概要
    \begin{coverabstract}
       The advent of quantum computing demands a rapid transition to post-quantum cryptographic solutions. In digital identity, SNARK-friendly schemes like Loquat\cite{cryptoeprint:2024/868} underpin post-quantum anonymous credential systems such as BDEC\cite{10.1007/978-981-96-0957-4_3}. However, BDEC’s reliance on static, custom zkSNARK\cite{cryptoeprint:2014/349} circuits for credential verification leads to critical inflexibility, rendering it impractical for dynamic attribute management. Zero-Knowledge Virtual Machines\cite{cryptoeprint:2023/1032} (zkVMs) promise a solution, offering to prove arbitrary programs and transform complex circuits into high-level code updates. This research will investigate the specific zero-knowledge properties of different zkVMs through comparative analysis of zkVMs and alternative SNARK circuit compilers\cite{lavin2024surveyapplicationszeroknowledgeproofs}, implementing and benchmarking the BDEC verifier within both approaches. This quantitative and qualitative analysis will determine which approach offers a more viable and agile foundation for the next generation of digital identity systems, specifically addressing the trade-offs between flexibility, performance with concrete metrics such as prover time, verification time, and memory usage.
        

    \end{coverabstract}
    \newpage
    
    %目次
    \tableofcontents
    \newpage
    
    %図目次
    \listoffigures
    \newpage
    
    %表目次
    \listoftables
    \newpage
    
    % 本文
    \input{01-intro}
    \newpage
    \section{Background}\label{sec: Background}
\subsection*{Notation}
Let $\lambda \in \mathbb{N}$ be the security parameter, $\mathsf{negl}(\lambda)$ denote a negligible function, and PPT stand for probabilistic polynomial-time algorithms.

\noindent We use $\mathsf{pp}$ to denote public parameters, and $sk$, $pk$ for secret and public keys, respectively. The symbol $\mathsf{crs}$ represents a common reference string typically used in zero-knowledge proof systems, and $pk$, $vk$ refer to the proving and verifying keys of zkSNARKs. The public input to a zkSNARK is denoted $x$, the private witness as $\omega$, and the proof as $\pi$. For digital signatures, $m$ is the message, and $\sigma$ is its signature. The relation or circuit verified by zkSNARKs is expressed as $C(x, \omega)$.

\noindent Regarding anonymous credentials, $\mathcal{A}$ denotes the universal set of attributes, with $\mathsf{attr}$ and $\mathsf{subattr}$ as subsets of $\mathcal{A}$. A credential is represented by $\mathsf{cred}$, and a shown credential by $\mathsf{show}$. The logical statement or predicate proved on attributes is $\mathsf{stmt}$, and auxiliary descriptions by $aux$.

\noindent For the Loquat post-quantum signature scheme, $\mathsf{L\text{-}\mathsf{pp}}$ denotes its specific public parameters, and $H$ the collision-resistant hash functions it uses. We denote $R1CS$ as the Rank-1 Constraint System representation of arithmetic circuits. The prime field used for the Legendre PRF is $\mathbb{F}_p$, and $\mathcal{L}(\cdot)$ the Legendre symbol pseudorandom function.


\subsection{Cryptographic Primitives} 
\subsubsection{zkSNARKs}
A zk-SNARK (more commonly referred to as Zero-Knowledge Succinct Non-Interactive Argument of Knowledge) \cite{ben2013snarks, groth2016efficient, sasson2014zerocash} is a cryptographic proof system that enables a \textit{prover} to convince a \textit{verifier} that they know a secret witness $\omega$ satisfying a publicly-known statement $C(x,\omega) = \mathsf{true}$, without revealing $\omega$, in a succinct and non-interactive format. Concretely, a zk-SNARK consists of the following algorithms,
\begin{enumerate}
    \item $\mathsf{Setup}(1^\lambda) \;\to\; (\mathsf{pk}, \mathsf{vk})$,
    \item $\mathsf{Prove}(\mathsf{pk}, x, \omega) \;\to\; \pi$,
    \item $\mathsf{Verify}(\mathsf{vk}, x, \pi) \;\in\; \{0,1\}$,
\end{enumerate}  
where $\lambda$ is the security parameter, $x$ is the public input and $\omega$ the private witness.

The zk-SNARK system should satisfy the following properties:
% if (x,w) satisfies the relation, the verifier accepts with overwhelming probability
\paragraph{\textbf{Completeness.}}  
If $C(x,\omega)=\mathsf{true}$, then for an honest \textit{prover}  
\begin{equation*}
\Pr\left[\mathsf{Verify}(\mathsf{vk}, x, \pi)=1 \;\Big|\; \pi \leftarrow \mathsf{Prove}(\mathsf{pk},x,\omega) \right] \ge 1-\mathsf{negl}(\lambda).
\end{equation*}

% the adversary tries to produce x,π such that verifier accepts, but no valid witness exists, or the prover doesn't know one
\paragraph{\textbf{Soundness.}}  
For any probabilistic polynomial-time adversary $\mathsf{Adv}$,
\begin{equation*}
    \Pr\Big[
    \mathsf{Verify}(\mathsf{vk},x,\pi)=1
    \;\wedge\;
    \neg \exists\,\omega'\!:\;C(x,\omega')=\mathsf{true}
    \;\Big|\;
    \begin{matrix}
    (\mathsf{pk},\mathsf{vk}) \leftarrow \mathsf{Setup}(1^\lambda),\\
    (x,\pi)\leftarrow \mathsf{Adv}(\mathsf{pk})
    \end{matrix}
    \Big]
    \le \mathsf{negl}(\lambda).
\end{equation*}

Moreover, there exists an extractor $\mathcal{E}$ such that if $\mathsf{Adv}$ outputs an accepting proof $(x,\pi)$ with non-negligible probability, then $\mathcal{E}(\mathsf{Adv}\text{'s state})$ outputs a valid $\omega'$ satisfying $C(x,\omega')=\mathsf{true}$.

% there exists a simulator Sim that, given the pp, can produce a simulated proof π* for any x satisfying the relation such that the distribution of (x,π*) is computationally indistinguishable from real (x,π) produced by honest prover
\paragraph{\textbf{Zero-Knowledge.}}  
There exists a simulator $\mathsf{Sim}$ that, given only the verification key $\mathsf{vk}$ and a public input $x$ with $C(x,\cdot)$ satisfiable, produces a proof $\pi^*$ such that the distribution 
$$(x,\pi) \;=\; (x, \pi \leftarrow \mathsf{Prove}(\mathsf{pk},x,\omega))$$ 
is computationally indistinguishable from $(x,\pi^*) \;=\; (x, \pi^* \leftarrow \mathsf{Sim}(\mathsf{vk},x))$.

% the proof size, prover runtime, and verifier runtime satisfy certain asymptotic bounds
\paragraph{\textbf{Succinctness.}}  
The size of the proof $\pi$ is short, typically $O(\mathsf{polylog}(|C|))$ or otherwise "sub-linear" in the size of the circuit representing $C$; the \textit{verifier}7s running time is similarly efficient (e.g., $O(|x| + \mathsf{polylog}(|C|))$).  

\subsubsection{Digital Signatures}
A digital signature scheme is composed of the following tuple of PPT algorithms $\mathsf{Setup}(1^\lambda)$, $\mathsf{KeyGen}(\mathsf{pp})$, $\mathsf{Sign}(sk,m)$, and $\mathsf{Verify}(pk,m,\sigma)$.

\begin{definition}[EUF-CMA Security]
A digital signature scheme \(\Sigma\) is existentially unforgeable under chosen-message attacks (EUF-CMA) if for all PPT adversaries $\mathcal{A}$ with access to a signing oracle $\mathcal{O}_{\mathsf{Sign}}$, the probability that $\mathcal{A}$ outputs a pair $(m^\star,\sigma^\star)$ such that $\mathsf{Verify}(\mathsf{pk},m^\star,\sigma^\star)=1$ and $m^\star$ was never queried to $\mathcal{O}_{\mathsf{Sign}}$ is negligible in $\lambda$.
\end{definition}

\noindent In this work, we instantiate the above notion with the \emph{Loquat} post-quantum signature scheme.

\begin{definition}[Loquat: A SNARK-Friendly Post-Quantum Signature]
    Loquat~\cite{loquat2024} is a digital signature scheme post-quantum secure under collision-resistant hashes and Legendre PRF, where
    \begin{align*}
    (\mathsf{L\text{-}pp}) &\leftarrow \mathsf{L\text{-}Setup}(1^\lambda),\\
    (sk, pk) &\leftarrow \mathsf{L\text{-}KeyGen}(\mathsf{L\text{-}pp}),\\
    \sigma &\leftarrow \mathsf{L\text{-}Sign}(\mathsf{L\text{-}pp}, sk, m),\\
    \{0,1\} &\leftarrow \mathsf{L\text{-}Verify}(pk, m, \sigma, \mathsf{L\text{-}pp}).
    \end{align*}
\end{definition}

\paragraph{Security.}
Loquat is proven EUF-CMA secure in the random-oracle model under the hardness of breaking the underlying Legendre PRF and the collision resistance of $H$~\cite{loquat2024}.

\noindent A crucial property for this work is that the Loquat verification algorithm admits an efficient rank-1 constraint system (R1CS) representation.

\paragraph{SNARK-friendliness.}
For the Loquat-128 parameter set instantiated with the Griffin hash function, the verification circuit can be represented using approximately $1.49\times 10^5$ R1CS constraints~\cite{loquat2024}. This is significantly smaller than known SNARK encodings of lattice-based post-quantum signature schemes such as CRYSTALS-Dilithium at comparable security levels~\cite{dilithium-spec}, and thus makes Loquat particularly suitable for use inside zkSNARK circuits.

\subsection{Anonymous Credentials}
\begin{definition}[Anonymous Credential System]
An anonymous credential system over an attribute universe \(\mathcal{A}\) is a tuple of PPT algorithms $\mathsf{AC.Setup}(1^\lambda)$, $\mathsf{AC.KeyGen}(\mathsf{pp})$, $\mathsf{AC.Issue}(isk, ipk, \mathsf{attr} \subseteq \mathcal{A}, aux)$, $\mathsf{AC.Show}(\mathsf{cred}, \mathsf{subattr} \subseteq \mathsf{attr}, \mathsf{stmt})$, and $\mathsf{AC.Verify}(\mathsf{pk_I}, \mathsf{show}, \pi, \mathsf{subattr}, \mathsf{stmt}, aux)$.
\end{definition}

\begin{definition}[BDEC: Post-Quantum Blockchain-based Digital Education Credential]
BDEC is a post-quantum anonymous credential system designed to securely and privately verify educational achievements on a blockchain. It builds upon generic anonymous credentials and ensures the following properties:
\begin{itemize}
    \item \textbf{Unforgeability}: No adversary can forge valid credentials.
    \item \textbf{Anonymity}: Credentials hide the user's identity.
    \item \textbf{Unlinkability}: Different proofs by the same user cannot be linked.
    \item \textbf{Conditional Linkability}: Selective linking enables managing fragmented learning records.
    \item \textbf{Revocation}: Credentials can be revoked if compromised.
\end{itemize}
\end{definition}

\subsubsection{Static Circuit Limitation}
BDEC fixes circuit size to maximum attributes $|A|\le N$ (e.g., $N=32$). Dynamic $|A|$ requires new $\mathsf{Setup}(N')$ per update, breaking efficiency.

\subsection{Dynamic Approaches}
\subsubsection{Zero-Knowledge Virtual Machines}

\subsubsection{SNARK Circuit Compilers}

    \newpage
    \section{Dynamism}\label{sec: Dynamism}
\begin{theorem}[Architectural Constraint Overhead for BDEC]\label{Architectural Constraint Overhead for BDEC}
    
\end{theorem}

\begin{theorem}[]\label{}
    
\end{theorem}

\begin{theorem}[]\label{}
    
\end{theorem}
    \newpage
    \input{05-evaluation}
    \newpage
    \input{06-discussion}
    \newpage
    \input{07-conclusion}
    \newpage
    \acknowledgement
    %消す 

    \newpage
    
    %  参考文献
    \bibliographystyle{jieeetran}
    \bibliography{ref}

    \newpage
    \appendix
\section{Proof of Theorem}
    
\end{document}
